\documentclass{article}
\usepackage{graphicx}
\usepackage{color}
\usepackage{amsmath}
\usepackage{fancyhdr}

\usepackage{url}
\usepackage{amsmath}
\usepackage{amssymb}
\usepackage{xspace}
\usepackage{graphicx}
\usepackage{hyperref}
\usepackage{listings}
\usepackage{tikz}
\usepackage{subcaption}
\usepackage[section]{placeins}

\sloppy
\definecolor{lightgray}{gray}{0.5}
\setlength{\parindent}{0pt}

\pagestyle{fancy}
\lhead{Page \thepage}

\renewcommand\thesubsection{\alph{subsection}.}
\renewcommand\thesubsubsection{\roman{subsubsection}.}




\title{HW5 \\
	\large CS 6210}
\date{12/16/2021}
\author{Spencer Peterson}

\pagestyle{fancy}
\rhead{Spencer Peterson}

\begin{document}
\maketitle

\section{Simpson's Rule}

\subsection{Implementation}

My implementation of Simpson's rule is shown below. 
\begin{verbatim}
function Area = simpsons_rule(F, a, b, N)
% Implementation of simpsons rule. F is a function handle, a and b are the
% starting and end points. N is the number of points to evaluate on. 
    width = (b-a)/N;

    points = width:width:b;

    Area = sum(F(points) .* width);

end
\end{verbatim}

\subsection{First Function}

The results for $f(x)  =x^p$ are shown in the table below.

\begin{center}
	\begin{tabular}{|c|c|c|c|c|c|c|c|c}
	\hline
p & Analytical Solution	& N=17 & N=33 & N=65 & N=129 & N=257 & N=513 \\ \hline
2 & 1/3=0.33333&	0.3633	&  0.3486&    0.3411  &  0.3372  & 0.3353  &  0.3343\\\hline
3 & 1/4=0.25	&	0.2803	&    0.2654 &   0.2578  &  0.2539&0.2519  &  0.2510\\ \hline
4 & 1/5=0.2&	0.2306	& 0.2155 &   0.2078 &   0.2039   & 0.2020&    0.2010\\ \hline
5 & 1/6=0.16667&	0.1975	&   0.1822 &   0.1745  &  0.1706 &   0.1686&    0.1676\\ \hline
6 & 1/7=0.14286&	0.1740	&   0.1585  & 0.1507  &  0.1468&    0.1448  &  0.1438\\ \hline
8 & 1/9=0.11111&	0.1428	&   0.1269  &  0.1190 &   0.1150  &  0.1131   & 0.1121 \\ \hline
	\end{tabular}
\end{center}

My implementation of Simpsons' rule doesn't start to converge until N is fairly large. Even then the error is still in the range of 1e-3. 

\subsection{Second Function}

I also tested it with the function $\int_0^{2\pi} 1 + \sin{x} \cdot \cos{\frac{2x}{3}} \cdot \sin{4x} \;dx$ The values start to converge once N is greater than 129.
\begin{center}
	\begin{tabular}{c|c|c|c|c|c|c}
N &		17 &	33 &	65 &	129 &	257 & 513\\ \hline
	Area &	6.30473283 &	6.30514394	& 6.30516885	& 6.30517044	& 6.30517054 &	6.30517055
	\end{tabular}	
\end{center}


\section{QuadTX}

The results for the QuadTX expirement are shown below. Overall increasing the tolerance by a factor of 1/10 tended to make the number of function calls grow by a factor of  1.3 and the time either by a similar but less precise amount.

\begin{center}
\begin{tabular}{c|c|c|c|c|c|c|c|c}
 Tolerance & 1e-07& 1e-08& 1e-09& 1e-10& 1e-11& 1e-12& 1e-13& 1e-14\\ \hline
Function Calls & 813& 1365& 2089& 3177& 5013& 7841& 12241& 19545\\ \hline
Time &  3.82 ms & 3.22 ms & 4.75 ms & 9.08 ms & 19.93 ms & 17.65 ms & 37.86 ms & 46.89 ms  \\
\end{tabular}
\end{center}


\section{Comparison of Methods}

We found that we needed 16777217 partitions to come up with an adequate answer with Simpson's method. This took 1.50 seconds. The QuadTX method took 19545 recursive calls and 45 ms.

\section{Coffee Cup}

\subsection{Analytical Solution}
\begin{align*}
	\frac{dT_c}{dt} &= -r(T_c-T_s)\\
	T_c &= Ae^{-rt} + T_s\\
	T_c(0) &= 84 = A+19\\
	A &= 65\\
	T_c &= 65 e^{-0.025 \cdot t} + 19
\end{align*}

\subsection{Forward Euler}

The plot of the Forward Euler method results are shown below:
\includegraphics{Figure1}


The error after the first step is shown in the table below:

\begin{center}
\begin{tabular}{c|c|c|c|c|c|c|c}
 h &  30& 15& 10& 5& 1& 0.5& 0.25\\ \hline
Error &  1.445e+01& 4.049e+00& 1.872e+00& 4.873e-01& 2.014e-02& 5.057e-03& 1.267e-03
\end{tabular}
\end{center}

The error seems to be linearly related to the step size $h$.

\section{ODE23 Implementation}
Below is my implementation of the ODE23 method.
\begin{verbatim}
function [t, y, error] = MyODE23(f, t0, tEnd, y0, h)
    % Implementation of ODE23 Method for problem 5 of homework 5 CS 6210. 
    
    curr_y = y0;
    curr_t = t0;
    ys = [curr_y];
    times = [curr_t];
    errors = [];
    
    while curr_t < tEnd
        S1 = f(curr_t, curr_y);

        S2 = f(curr_t + h/2, curr_y + h/2 * S1);

        S3 = f(curr_t + 3*h/4, curr_y + 3 * h/ 4 * S2);

        next_t = curr_t + h;

        next_y = curr_y + h/9 * (2 * S1 + 3 * S2 + 4 * S3);


        S4 = f(next_t, next_y);
        curr_error = h/72 * (-5 * S1 + 6 * S2 + 8*S3 - 9 * S4);

        times = [times next_t];
        ys = [ys next_y];

        errors = [errors curr_error];
        curr_t = next_t;
        curr_y = next_y;

    end
    t = times;
    y = ys;
    error = errors;
end
\end{verbatim}

\section{ODE23 Plot}

\includegraphics{Figure2}

The predicted and actual errors for the first time step are shown below:

\begin{center}
\begin{tabular}{c|c|c|c|c|c|c|c}
 h &  30& 15& 10& 5& 1& 0.5& 0.25\\ \hline
Predicted Error & 6.583e-02& 3.064e-02& 1.236e-02& 2.042e-03& 2.012e-05& 2.579e-06& 3.265e-07\\
Actual Error &7.429e-01& 4.978e-02& 1.007e-02& 6.450e-04& 1.053e-06& 6.596e-08& 4.127e-09
\end{tabular}
\end{center}

The actual error is larger than predicted for a large h and smaller than predicted for a small h.

\section{Larger r}


The method does blow up for certain values of h. The error estimator also blows up when a solution blows up.

\includegraphics{Figure3}

\section{Stormer Verlet Method}

The results of the Stormer Verlet method are shown below. I used a time step of 1e-3 seconds for all trials.
\includegraphics{Figure5}

I measured the error in the Hamiltonian by finding the infinity norm of the difference of the analytical Hamiltonian and the calculated Hamiltonian. The result was 6.250. This is within the same order of magnitude that we got in class.

\section{Trapezoidal and Backward Euler Methds}


I implemented the Reverse Euler as well using an inversion of a 2x2 matrix. The derivation of this matrix is shown below:

\begin{align*}
x(t+1) &= x(t) + dt * v(t)\\
v(t+1) &= v(t) - dt  \omega^2 x(t)\\
\\
x(t+1) -  dt * v(t) &= x(t)\\
dt  \omega^2 x(t) + v(t+1) &= v(t) \\
\\
\begin{bmatrix}
1 & -dt\\
dt\omega^2 & 1 
\end{bmatrix}
\begin{bmatrix}
x(t+1) \\ v(t+1)
\end{bmatrix}
&= \begin{bmatrix}
x(t) \\ v(t)
\end{bmatrix}\\
A
\begin{bmatrix}
x(t+1) \\ v(t+1)
\end{bmatrix}
&= \begin{bmatrix}
x(t) \\ v(t)
\end{bmatrix}\\
\begin{bmatrix}
x(t+1) \\ v(t+1)
\end{bmatrix}
&= A^{-1} \begin{bmatrix}
x(t) \\ v(t)
\end{bmatrix}
\end{align*}

The code was implemented as follows. Please see code for the full context as this is just the matrix math as it relates to what was determined above.

\begin{verbatim}
A = [1 -dt; dt*omega^2 1]; % Matrix to be inverted for implicit Euler's
Ainv = inv(A);

temp = Ainv * [x_re(step); v_re(step)];
x_re(step+1) = temp(1);
v_re(step+1) = temp(2);
\end{verbatim}

 The results are shown below:

\includegraphics{Figure6}

The Hamiltonian is not preserved and using the same method as above for the Stormer Verlet method, the erorr was 2.5e3.



Additionally, I implemented the Trapezoidal method. The derivation for the matrices used is shown below:

\begin{align*}
x(t+1) &= x(t) +\frac{dt}{2}( v(t) + v(t+1))\\
v(t+1) &= v(t) +\frac{dt}{2} (  -\omega^2 x(t) -\omega^2 x(t+1))\\
\\
x(t+1) - \frac{dt}{2} v(t+1)&= x(t) +\frac{dt}{2} v(t)\\
\frac{dt}{2} \omega^2 x(t+1) +  v(t+1) &= -\frac{dt}{2}\omega^2 x(t) +v(t) \\
\\
\begin{bmatrix}
1 & -\frac{dt}{2} \\ \frac{dt}{2}\omega^2& 1
\end{bmatrix}
\begin{bmatrix}
x(t+1) \\ v(t+1)
\end{bmatrix}
&= \begin{bmatrix}
1 & \frac{dt}{2} \\ -\frac{dt}{2}\omega^2  & 1
\end{bmatrix}
\begin{bmatrix}
x(t) \\ v(t)
\end{bmatrix}\\
\\
B
\begin{bmatrix}
x(t+1) \\ v(t+1)
\end{bmatrix}
&= C
\begin{bmatrix}
x(t) \\ v(t)
\end{bmatrix}\\
\\
\begin{bmatrix}
x(t+1) \\ v(t+1)
\end{bmatrix}
&= B^{-1}C
\begin{bmatrix}
x(t) \\ v(t)
\end{bmatrix}
\end{align*}

The matrices B and C were calculated as follows for the implementation of this method.

\begin{verbatim}
Binv = inv([1 -1/2*dt; 1/2*dt*omega^2 1]);
C = [1 1/2*dt; -1/2*dt*omega^2 1];
\end{verbatim}

And they were used to calculate the next step as follows:
\begin{verbatim}
temp = Binv * C * [x_tr(step); v_tr(step)];
x_tr(step+1) = temp(1);
v_tr(step+1) = temp(2);
\end{verbatim}

The results for this method are shown below:

\includegraphics{Figure7}


The error for the Hamiltonian was calculated as 1.55e-9. This is much smaller than any of the other methods we've tried in class. 







\end{document}